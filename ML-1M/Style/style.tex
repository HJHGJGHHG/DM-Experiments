\documentclass[a4paper,12pt]{article}
\usepackage[fontset=none]{ctex}
\usepackage{fancyhdr}
\usepackage{wrapfig}

% 数学公式和化学方程式
\usepackage{amsmath, amsthm, amssymb, amsfonts}
\usepackage{latexsym,bm}
\usepackage[version=4]{mhchem}
\makeatletter
\newcommand{\rmnum}[1]{\romannumeral #1}%小写罗马数字
\newcommand{\Rmnum}[1]{\expandafter\@slowromancap\romannumeral #1@}%大写罗马数字
\makeatother
\renewcommand{\frac}{\dfrac}
\renewcommand{\leq}{\leqslant}
\renewcommand{\geq}{\geqslant}
\renewcommand{\le}{\leqslant}
\renewcommand{\ge}{\geqslant}
\renewcommand{\epsilon}{\varepsilon}

\newcommand{\tran}{^T}
\newcommand{\conj}[1]{{\overline{#1}}}
\newcommand{\hermconj}{^H}

\newcommand{\va}{{\bm{a}}}
\newcommand{\vb}{{\bm{b}}}
\newcommand{\vc}{{\bm{c}}}
\newcommand{\vd}{{\bm{d}}}
\newcommand{\ve}{{\bm{e}}}
\newcommand{\vf}{{\bm{f}}}
\newcommand{\vg}{{\bm{g}}}
\newcommand{\vh}{{\bm{h}}}
\newcommand{\vi}{{\bm{i}}}
\newcommand{\vj}{{\bm{j}}}
\newcommand{\vk}{{\bm{k}}}
\newcommand{\vl}{{\bm{l}}}
\newcommand{\vm}{{\bm{m}}}
\newcommand{\vn}{{\bm{n}}}
\newcommand{\vo}{{\bm{o}}}
\newcommand{\vp}{{\bm{p}}}
\newcommand{\vq}{{\bm{q}}}
\newcommand{\vr}{{\bm{r}}}
\newcommand{\vs}{{\bm{s}}}
\newcommand{\vt}{{\bm{t}}}
\newcommand{\vu}{{\bm{u}}}
\newcommand{\vv}{{\bm{v}}}
\newcommand{\vw}{{\bm{w}}}
\newcommand{\vx}{{\bm{x}}}
\newcommand{\vy}{{\bm{y}}}
\newcommand{\vz}{{\bm{z}}}
\newcommand{\vzero}{{\bm{0}}}

\newcommand{\vepsilon}{{\bm{\epsilon}}}
\newcommand{\vtheta}{{\bm{\theta}}}
\newcommand{\vpsi}{{\bm{\psi}}}
\newcommand{\vpi}{{\bm{\pi}}}
\newcommand{\vphi}{{\bm{\phi}}}

\newcommand\bigO{{\mathcal{O}}}

\newcommand{\vA}{{\bm{A}}}
\newcommand{\vB}{{\bm{B}}}
\newcommand{\vC}{{\bm{C}}}
\newcommand{\vD}{{\bm{D}}}
\newcommand{\vE}{{\bm{E}}}
\newcommand{\vF}{{\bm{F}}}
\newcommand{\vG}{{\bm{G}}}
\newcommand{\vH}{{\bm{H}}}
\newcommand{\vI}{{\bm{I}}}
\newcommand{\vJ}{{\bm{J}}}
\newcommand{\vK}{{\bm{K}}}
\newcommand{\vL}{{\bm{L}}}
\newcommand{\vM}{{\bm{M}}}
\newcommand{\vN}{{\bm{N}}}
\newcommand{\vO}{{\bm{O}}}
\newcommand{\vP}{{\bm{P}}}
\newcommand{\vQ}{{\bm{Q}}}
\newcommand{\vR}{{\bm{R}}}
\newcommand{\vS}{{\bm{S}}}
\newcommand{\vT}{{\bm{T}}}
\newcommand{\vU}{{\bm{U}}}
\newcommand{\vV}{{\bm{V}}}
\newcommand{\vW}{{\bm{W}}}
\newcommand{\vX}{{\bm{X}}}
\newcommand{\vY}{{\bm{Y}}}
\newcommand{\vZ}{{\bm{Z}}}

\newcommand{\ones}[1]{{\bm{1}_{#1}}}
\newcommand{\zeros}[1]{{\bm{0}_{#1}}}
\newcommand{\eye}[1]{{\bm{E}_{#1}}}

\newcommand{\vect}[1]{{\bm{#1}}}
\newcommand{\mat}[1]{{\bm{#1}}}

\renewcommand{\d}{{\mathrm{d}}}
\newcommand{\pder}[2]{{\frac{\partial #1}{\partial #2}}}
\newcommand{\oder}[2]{{\frac{\mathrm{d} #1}{\mathrm{d} #2}}}
\newcommand{\npder}[2]{{\nicefrac{\partial #1}{\partial #2}}}
\newcommand{\popt}[2]{{\frac{\partial}{\partial #2}#1}}
\newcommand{\oopt}[2]{{\frac{\mathrm{d}}{\mathrm{d} #2}#1}}
\newcommand{\inte}[4]{{\int_{#1}^{#2}#3\mathrm{d}#4}}

\newcommand{\argmin}{{\operatornamewithlimits{argmin}}}
\newcommand{\argmax}{{\operatornamewithlimits{argmax}}}
\newcommand{\tr}{{\mathrm{tr}}}
\newcommand{\Prob}{{\mathrm{P}}}
\newcommand{\E}{{\mathrm{E}}}
\newcommand{\D}{{\mathrm{D}}}

\newcommand{\<}{{\langle}}
\renewcommand{\>}{{\rangle}}



% 插入PDF文件
\usepackage[final]{pdfpages}

% 引文使用
\usepackage[backend=biber,style=gb7714-2015,gbpub=false]
{biblatex}%align=gb7714-2015
\addbibresource[location=local]{Ref/Collection.bib}
% \usepackage[colorlinks=true,pdfstartview=FitH,%
% linkcolor=blue,anchorcolor=violet,citecolor=magenta]{hyperref}%加载hyperref宏包,使用超链接

% 插图使用
\usepackage{graphicx}
\usepackage{float}
% \usepackage{epstopdf}
\usepackage{caption}
\usepackage{subfigure}  %插入多图时用子图显示的宏包

% 表格使用
\usepackage{bigstrut}
\usepackage{booktabs}
\usepackage{multicol}
\usepackage{multirow}

% LaTeX距离设置:
% mm	毫米	1 mm = 2.845 pt
% pt	点		1 pt = 0.351 mm
% bp	大点	1 bp = 0.353 mm > 1 pt
% dd	迪多	1 dd = 0.376 mm = 1.07 pt
% pc	排卡	1 pc = 4.218 mm = 12 pt
% sp	定标点	65536 sp = 1 pt
% cm	厘米	1 cm= 10 mm= 28.453 pt
% cc	西塞罗	1 cc= 4.513 mm= 12 dd = 12.84 pt
% in	英寸	1 in = 25.4 mm = 72.27 pt
% ex	ex	1 ex = 当前字体尺寸中 x 的高度
% em	em	1 em = 当前字体尺寸中 M 的宽度

% 字体大小:
\newcommand{\chuhao}{\fontsize{42.2pt}{\baselineskip}\selectfont}
\newcommand{\xiaochu}{\fontsize{36.1pt}{\baselineskip}\selectfont}
\newcommand{\yihao}{\fontsize{26.1pt}{\baselineskip}\selectfont}
\newcommand{\xiaoyi}{\fontsize{24.1pt}{\baselineskip}\selectfont}
\newcommand{\erhao}{\fontsize{22.1pt}{\baselineskip}\selectfont}
\newcommand{\xiaoer}{\fontsize{18.1pt}{\baselineskip}\selectfont}
\newcommand{\sanhao}{\fontsize{16.1pt}{\baselineskip}\selectfont}
\newcommand{\xiaosan}{\fontsize{15.1pt}{\baselineskip}\selectfont}
\newcommand{\sihao}{\fontsize{14.1pt}{\baselineskip}\selectfont}
\newcommand{\xiaosi}{\fontsize{12.1pt}{\baselineskip}\selectfont}
\newcommand{\wuhao}{\fontsize{10.5pt}{\baselineskip}\selectfont}
\newcommand{\xiaowu}{\fontsize{9.0pt}{\baselineskip}\selectfont}
\newcommand{\liuhao}{\fontsize{7.5pt}{\baselineskip}\selectfont}
\newcommand{\xiaoliu}{\fontsize{6.5pt}{\baselineskip}\selectfont}
\newcommand{\qihao}{\fontsize{5.5pt}{\baselineskip}\selectfont}
\newcommand{\bahao}{\fontsize{5pt}{\baselineskip}\selectfont}

% 字体
% \usepackage{fontspec, xunicode, xltxtra}
% \usepackage{xeCJK}
\setmainfont{Times New Roman}
\setsansfont{Droid Sans}
\setmonofont{Courier New}
\setCJKmainfont[Path="Style/",AutoFakeBold,ItalicFont=simkai.ttf]{simsun.ttc}
\setCJKsansfont[Path="Style/",AutoFakeBold]{simhei.ttf}
\setCJKmonofont[Path="Style/",AutoFakeBold]{simfang.ttf}
% 需要的话
%\setCJKfamilyfont{zhfs}[AutoFakeBold]{simfang.ttc}
%\setCJKfamilyfont{zhkai}[AutoFakeBold]{simkai.ttf}
%\setCJKfamilyfont{zhsong}[AutoFakeBold]{simsun.ttc}
\setCJKfamilyfont{zhhei}[AutoFakeBold]{simhei.ttf}

\newcommand{\heiti}{\CJKfamily{zhhei}}


\usepackage{titlesec}
%section等前后间距
\titlespacing*{\section}{0pt}{1ex plus .0ex minus .0ex}{1ex plus .0ex}
\titlespacing*{\subsection}{0pt}{0ex plus .0ex minus .0ex}{0ex plus .0ex}
%section等字体设置和字号大小
\titleformat*{\section}{\large\bfseries}
\titleformat*{\subsection}{\normalsize\bfseries}

%纸张和页边距
\usepackage[a4paper,left=1.5cm,right=1.5cm,top=1.5cm,bottom=1.5cm]{geometry}

%行间距
\usepackage{setspace}
\renewcommand{\baselinestretch}{1.5}
\AtBeginDocument{
    \setlength{\abovedisplayskip}{1.5pt}%公式前后间距,强行在导言区中设置
    \setlength{\belowdisplayskip}{1.5pt}%
    \setlength{\abovedisplayshortskip}{1.5pt}%
    \setlength{\belowdisplayshortskip}{1.5pt}%
    %图片和浮动对象间距
    \setlength{\floatsep}{2pt plus 2pt minus 2pt}%出现在页面的顶部或底部的浮动对象之间的垂直距离。 缺省为 12pt plus 2pt minus 2pt
    \setlength{\textfloatsep}{2pt plus 2pt minus 2pt}%出现在页面的顶部或底部的浮动对象与文本之间的垂直距离。缺省为 20pt plus 2pt minus 4pt
    \setlength{\intextsep}{2pt plus 2pt minus 2pt}%出现在页面中间的浮动对象(如使用了 h 选项 的浮动对象)与上下方文本之间的垂直距离。 缺省为 12pt plus 2pt minus 2pt
}
% 间距有关变量:
% \baselineskip:行基线间距。
% \lineskip:行间距。
% \baselinestretch:伸展因子。
% \parskip:部分段间距。
% \lineskiplimit:当两行字之间的距离小于\lineskiplimit时,行距自动设为\lineskip。
% 段间距:\lineskip + \parskip
% 行间距:\lineskip = \baselineskip * \baselinestretch

%首行缩进
\usepackage{indentfirst}
\setlength{\parindent}{2pt}

%文本和图片等浮动对象占比
\renewcommand{\floatpagefraction}{.85}%浮动页中浮动对象最大占比
\renewcommand{\textfraction}{.15}%文本页中文本最小占比
\renewcommand{\topfraction}{.65}%页面顶部可以用来放置浮动对象的高度与整个页面高度的最大比例
\renewcommand{\bottomfraction}{.60}

%符号列表间距设置
% \setlength{\leftmargin}{1.2em}%左边界
% \setlength{\parsep}{0ex}%段落间距
% \setlength{\topsep}{1ex}%列表到上下文的垂直距离
% \setlength{\itemsep}{0.5ex}%条目间距
% \setlength{\labelsep}{0.3em}%标号和列表项之间的距离,默认0.5em
% \setlength{\itemindent}{1.1em}%标签缩进量
% \setlength{\listparindent}{0em} %段落缩进量

% 列表格式
\usepackage{enumitem}
\setenumerate{itemsep=0pt,partopsep=0pt,parsep=\parskip,topsep=5pt,itemindent=2em}
\setitemize{itemsep=0pt,partopsep=0pt,parsep=\parskip,topsep=5pt,itemindent=2em}
\setdescription{itemsep=0pt,partopsep=0pt,parsep=\parskip,topsep=5pt,itemindent=2em}

% 页码设置
\pagestyle{empty}

% 水印
\usepackage{eso-pic}
\newcommand\BackgroundPicture{%
  \put(0,0){%
    \parbox[b][\paperheight]{\paperwidth}{%
      \vfill
      \centering%
\begin{tikzpicture}[remember picture,overlay]
\node [rotate=45,scale=6,text opacity=0.3] at (current page.center) {水印样例}; %中括号内是旋转角度,字体大小
\end{tikzpicture}%
      \vfill
    }}}

% 标题样式
\newcommand\subtitle[1]{{\Large #1}} % 定义副标题
\makeatletter % change default title style
\renewcommand*\maketitle{%
    \begin{center}% 居中标题
        % \bfseries % 默认粗体
        {\LARGE\bfseries\@title \par} % LARGE字号,默认粗体
        \vskip 1em% %%%  标题下面只有1em的缩进或margin
        {\@author \par}
        % {\global\let\author\@empty}%
        {\global\let\date\@empty}%
        \thispagestyle{empty} %  不设置页面样式
    \end{center}%
  \setcounter{footnote}{0}%
}
\makeatother

% 中文摘要
\newenvironment{cnabstract}{%
  \par\sihao
  \noindent\mbox{}\hfill{\bfseries\textsf \cnabstractname}\hfill\mbox{}\par
  \vskip 2.5ex}{\par\vskip 2.5ex}
\newcommand{\cnabstractname}{摘 要}

% 页眉页脚
\pagestyle{fancy}
\fancypagestyle{preContent}{
    \fancyhead{}
    \renewcommand\headrulewidth{0pt}
    \fancyfoot[C]{\thepage}
}

% 插入代码设置
\usepackage{listings}
\lstset{
 columns=fixed,       
 numbers=left,                                        % 在左侧显示行号
 numberstyle=\tiny\color{gray},                       % 设定行号格式
 frame=none,                                          % 不显示背景边框
 backgroundcolor=\color[RGB]{245,245,244},            % 设定背景颜色
 keywordstyle=\color[RGB]{40,40,255},                 % 设定关键字颜色
 numberstyle=\footnotesize\color{darkgray},           
 commentstyle=\rmfamily\it\color[RGB]{0,96,96},                % 设置代码注释的格式
 stringstyle=\rmfamily\slshape\color[RGB]{128,0,0},   % 设置字符串格式
 showstringspaces=false,                              % 不显示字符串中的空格
 language=Python,                                        % 设置语言
 breaklines      =   true,
}
